\section{Discussion}

\para{Multiple application classes:} Sometimes, system administrators need to
use multiple lossless priorities to keep different traffic classes from impacting each
other. For example, in~\cite{dcqcn} congestion notification packets were
assigned a separate lossless class from data traffic to ensure that they would
not just be delivered losslessly, but also would not be held up behind data
traffic.

A n{\"a}ive way to use \sysname{} in such cases is to treat each application (or
traffic class) separately.  For example, in \S\ref{subsec:combine}, we showed
that for the Clos network, if we want to make  all paths with no more than $M$
bounces lossless, we need $M+1$ priorities. If there are $N$ applications, the
n{\"a}ive approach would use $N*(M+1)$ priorities.  However, the switches may
not have sufficient buffer to support this large number of lossless queues
(\S\ref{subsec:pfcheadroom}.

However, if we are willing to trade off some isolation, we can proceed as
follows.  We start the first lossless class with tag 1, and uses tag upto $M+1$.
The second lossless class starts with tag 2, and change tags in the same way as
the first class.  That is, whenever it has a bounce, the tag will increase by
one. The second lossless class uses tag from 2 to $M+2$. This goes on until the
$N'th$ lossless class. The total number of tags and lossless classes required is
$M + N$. The operator can further reduce the number of tags required by allowing
some application classes to tolerate fewer bounces than others.

We can prove that there is still no deadlock after such mix, by revisiting two
properties described in \S\ref{subsec:specific_deadlock_free}. First, there is
still no deadlock within each tag, because each tag is still a set of
``up-down'' routing. Second, the update of tags is still monotonical. We omit
formal proof for brevity. 

The reduced isolation may be quite acceptable, since only a small
fraction of packets experience one-bounce and may mix with traffic in the next
lossless class. 

This technique can be generalized for the output of Algorithm~\ref{alg:greedy}.
We omit details.

\para{The need to specify expected lossless paths:} The need to specify expected
lossless paths is not a problem in practice. For Clos networks, it is easy to
enumerate paths with any given limit on bouncing. For a general topology, as long
as the routing is traffic agnostic, it is usually easy to determine what routes
the routing algorithm will compute -- e.g. BGP will find shortest AS path etc.
In case an SDN controller is used, the controller algorithm can be used to
generate the paths under a variety of conditions. Even for traffic aware
routing, the operator can run a variety of simulations. 

The reader may also dislike the fact that we may eventually push a wayward
packet into a lossy queue. We stress that we do this only as a last resort, and
we reiterate that it does not mean that the packets are automatically or
immediately dropped. 

\para{Dynamic topologies:} Recently, a number of researches have proposed
changing the topology of the network in response to traffic patterns, using
optical or wireless links~\cite{helios,flyways,projector,mirrormirror}.
\sysname{} cannot be readily used for dynamic topologies, since we would need to
recompute tags every time the topology changes. To the best of our knowledge, no
modern data centers currently use dynamic topologies. Still, the problem of
recomputing tags on-the-fly is an interesting research challenge, and we plan to
work on it in future.

\para{PFC alternatives:} The reader may wonder if PFC is worth all the trouble
it causes. We are sympathetic to this view, and are actively investigating
alternative schemes, including minimizing PFC generation (e.g.  using
DCQCN~\cite{dcqcn} or Timely~\cite{timely}, better retransmission schemes for IB
transport layer~\cite{xxx}, as well as other more novel schemes.  It is not yet
clear to us that need for something akin to PFC can be completely obviated for
high performance RDMA networks. More importantly, we must ensure proper
operation of currently deployed RDMA hardware, which is reliant on PFC, and in
which we and our customers have made substantial investments.

