\section{Discussion}

\para{Multiple application classes:} Sometimes, system administrators need to
use multiple lossless priorities to keep different traffic classes from
impacting each other. For example, in~\cite{dcqcn} congestion notification
packets were assigned a separate lossless class to ensure that these packets
would not be held up by data traffic.

A n{\"a}ive way to use \sysname{} in such cases is to treat each application (or
traffic class) separately.  For example, in \S\ref{subsec:combine}, we showed
that for the Clos network, if ELP contained paths with no more than $M$ bounces
lossless, we need $M+1$ priorities. If there are $N$ applications, the n{\"a}ive
approach would use $N*(M+1)$ priorities. 

However, we can use fewer priorities by trading off some isolation.  The first
application class starts with tag 1, and uses tags upto $M+1$.  The second
application class starts with tag 2, and also increases tags by 1 at each
bounce.  Thus, the second application class uses tags $2 \ldots M+2$. Thus, for
$N$ application classes need just $M + N -1$ tags. This can be further reduced
by making some application classes to tolerate fewer bounces than others.

Note that there is still no deadlock after such mix. First, there is still no
deadlock within each tag, because each tag is still a set of ``up-down''
routing. Second, the update of tags is still monotonic. We omit formal proof for
brevity.

The reduced isolation may be acceptable, since only a small fraction of packets
experience one-bounce and may mix with traffic in the next lossless class.  This
technique can be generalized for the output of Algorithm~\ref{alg:greedy}.

\para{Specifying ELP:} The need to specify expected lossless paths is not a
problem in practice. For Clos networks, it is easy to enumerate paths with any
given limit on bouncing. In general, as long as routing is traffic agnostic, it
is usually easy to determine what routes the routing algorithm will compute --
e.g. BGP will find shortest AS path etc.  If an SDN controller is used, the
controller algorithm can be used to generate the paths under a variety of
simulated conditions. ECMP is handled by including all possible paths.

We stress again that there are no restrictions on routes included in ELP, apart
from the common-sense requirement that each route be loop-free. Once ELP is
specified, we can handle any subsequent abnormalities.

\para{Use of lossy queue:} Some may dislike the fact that we may eventually push
a wayward packet into a lossy queue. We stress that we do this only as a last
resort, and we reiterate that it does not mean that the packets are
automatically or immediately dropped.

\para{Topology changes:} \sysname{} has to generate a new set of tags if ELP is
updated.  ELP is typically updated when switches or links are added to the
network. If a FatTree-like topology is expanded by adding new ``pods'' under
existing spines (i.e. by using up empty ports on spine switches), none of the
older switches need any rule changes.  Jellyfish-like random topologies may need
more extensive changes.

Note that switch and link failures are common in data center
networks~\cite{netpilot}, and we have have shown
(Figure~\ref{fig:clos_1_bounce}) that \sysname{} handles them fine. 

\para{Flexible topology architectures:} \sysname{} can support architectures
like Helios~\cite{helios}, Flyways~\cite{flyways} or Projector~\cite{projector},
as long as ELP set is specified.

\para{PFC alternatives:} One might argue that PFC is not worth the trouble it
causes; and we should focus on getting rid of PFC altogether.  We are
sympathetic to this view, and are actively investigating numerous schemes,
including minimizing PFC generation (e.g.  using DCQCN~\cite{dcqcn} or
Timely~\cite{timely}, better retransmission schemes in the NIC, as well as other
more novel schemes.  For now, we need \sysname{} to ensure proper operation of
currently deployed RDMA hardware, which is reliant on PFC, and in which we and
others have made substantial investments.

