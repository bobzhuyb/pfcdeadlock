\section{Solution: Tag-based System}\label{sec:generic}

We explore the practical solution for preventing PFC deadlock without any specific assumptions on the topology and routing that operator chooses.

\subsection{Key Ideas}

\begin{enumerate}
	\item Our first insight is to change the lossless queue of a packet right before it may cause cyclic buffer dependency if it does not change priority.
	
	\item Each switch must make the decision for each packet locally and distributedly, based on the packet's past path. This operation should be purely done in the
	data plane.

	\item The way the switch gets a packet's past path is by the tag the packet carries. We encode the information of past path into the tag, and let switches change 
	the tag along the path. 

\end{enumerate}

The overhead of this design is that we will need {\em multiple} switch lossless queues to support one lossless application class.
This overhead is inevitable given that routing is not changed. 
As the first step, we focus on supporting {\em just one} lossless application class. In Section~\ref{sec:specific}, we will revisit 
this issue and show how we may support multiple application classes efficiently.


\para{To use multiple switch lossless queues for one application class}, we divide the buffer of network nodes into $k$ partitions, and let the $j$-$th$ partition associated with priority class $j$. If a packet is classified into priority class $j$, it will  be buffered in the $j$-$th$ buffer partition. Packets queued in queues $q_{in}^{i,j}$ and $q_{out}^{i,j}$ ($1\leq i \leq n$) are the packets currently buffered in the $j$-$th$ buffer partition.

% Let $d$ be the number of hops of the longest legal routing path in the network, $n$ be the number of switch ports, and $m_{p}$ be the maximum packet size. Our buffer division will always ensure $k \geq d$, and the size of any buffer partition is no smaller than $n*m_{p}$. 

\subsection{Tagging System}

The tag of the packet must carry the information of each packet's past path. Thus, we divide the path of each packet into multiple stages,
and use the tag to represent the stage each packet is on. We can later prove that the network is deadlock-free if the tagging system satisfies
the two requirements below:

\begin{enumerate}

	\item A packet's tag is unchanged or changed at each hop of its path. When the tag is changed, it must be changed monotonically along any packet’s path (e.g., always increasing).

	\item All packets that have the same tag cannot have cyclic buffer dependency.

\end{enumerate}

On each switch, packets are classified into different priority classes purely based on the tags. Then we can prove that our network is deadlock-free.


\para{Proof of deadlock-free} \fixme{Need to rewrite the following.}

\textbf{Claim:} our TTL-based solution is deadlock-free regardless of the packet scheduling algorithms.

\textbf{Proof:} To prove our TTL-based solution is deadlock-free, we prove by contradiction that no legal buffer state can be deadlocked buffer state under our TTL-based solution.

Assuming there exists a legal buffer state  $BS_N(t)$ which is also a deadlocked buffer state under our TTL-based solution. If no new packets are injected into the network since $t$, according to Equation~(\ref{eqn:deadlockstatedef}), $BS_N(t)$ will converge into a fixed non-empty buffer state $BS_N(t_0)$ at some finite time $t_0>t$. 


\textbf{Case 1:} All the VEQs in $BS_N(t_0)$ is empty. As $BS_N(t_0) \neq BS^0_N$, $BS_N(t_0)$ has at least one non-empty VIQ. As $BS_N(t)$ is a legal buffer state, accroding to Equation~(\ref{eqn:legalstatecon}), there are finite number of packets queued in non-empty VIQs of $BS_N(t_0)$. Then according to Equations~(\ref{eqn:schecon1}), (\ref{eqn:nodupschedule}) and (\ref{eqn:fwdcon}), any unscheduled packet remaining in any VIQ will be forwarded to some VEQ within finite time at some finite time $t_2>t_0$. This means that $BS_N(t_0)$ will transition to some other buffer state, which violates the fact that $\forall t_1>t_0, BS_N(t_1)\equiv BS_N(t_0)$.

\textbf{Case 2:} There exist some non-empty VEQs (at least one) in $BS_N(t_0)$. Let $q_{out}^{i,m}$ be the queue of highest priority class among all the non-empty VEQ queues ($m\leq k$). Under our TTL-based solution, packets in $q_{out}^{i,m}$ will not be paused by PFC PAUSE messages as packets can only be paused by packets of higher priority class. According to Equations~(\ref{eqn:schecon2}) and (\ref{eqn:transcon}), packets in $q_{out}^{i,m}$ will be transmitted to next hop within finite time. This means that $BS_N(t_0)$ will transition to some other buffer state  at some finite time $t_2>t_0$, which violates the fact that $\forall t_1>t_0, BS_N(t_1)\equiv BS_N(t_0)$.

Based on the above discussion, the assumption we made will cause contradiction in both cases. Hence $BS_N(t)$ is not a deadlocked buffer state under our TTL-based solution. So our TTL-based solution is deadlock-free regardless of the packet scheduling algorithms.


\para{TTL-based tagging system.} In a generic topology and routing, TTL is a natural tagging system that satisfies our requirements.


\textbf{TTL-based packet buffering}: 
\begin{enumerate}
	\item  Initially, we set the TTL values of all packets to $ttl_0=d$ at all the source servers. TTL value of every packet will be decreased by 1 per hop. At all the source servers, packets are buffered in a buffer of priority class $0$.
	
	\item Let $ttl_i$ be the TTL value of a packet $p$ at its $i$-$th$ hop ($i \geq 0$). If $ttl_i = 0$, packet $p$ will be dropped by the receiving switch or server. At every hop, the priority class of any incoming packet $p$ is calculated as $\lambda_p = ttl_0 - ttl_i $. Packet $p$ will be buffered and queued according to the calculated priority class $\lambda_p$ at every hop.
\end{enumerate}
  
The drawback of TTL-based tagging is that it requires a lot of lossless queues. 


%\subsubsection{Proof of Deadlock-free Property}\label{subsec:proof}
%In this part, we are going to prove that our TTL-based solution is deadlock-free regardless of the packet scheduling algorithms.

%In our proof, we use $p_{size}$, $p_{ttl}$ and $p_{dst}$ to denote the size, the TTL value and the destination of a packet $p$, respectively. Any ingress queue $q_{in}^{i,j}$ and any egress queue $q_{out}^{i,j}$ are viewed as packet sets. According to the switch model, we have $q_{in}^{i}=\cup_{j=1}^{k}q_{in}^{i,j}$, and $q_{out}^{i}=\cup_{j=1}^{k}q_{out}^{i,j}$.

%We use $|q_{in}^{i,j}|$ and $|q_{out}^{i,j}|$ to denote the queue lengths of $q_{in}^{i,j}$ and $q_{out}^{i,j}$, where $|q_{in}^{i,j}|=\sum_{p\in q_{in}^{i,j}}p_{size}$, and $|q_{out}^{i,j}|=\sum_{p\in q_{out}^{i,j}}p_{size}$.


\subsection{Greedy Algorithm for Minimizing the Number of Lossless Queues}

\fixme{Here is the generic algorithm in ppt.}