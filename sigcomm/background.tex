\section{Background}
\label{sec:background}
{\bf RDMA and RoCE:} Remote Direct Memory Access (RDMA) technology offers high
throughput, low latency and low CPU overhead, by bypassing end-host networking
stacks. Instead, Network Interface Cards(NICS) transfer data in and out of
pre-registered memory buffers at the two end hosts.  In modern data centers,
RDMA is deployed using RDMA over Converged Ethernet V2 (RoCE)
standard~\cite{roce,rroce}

{\bf PFC:} RoCE needs a lossless L2 layer for optimal performance. This is
accomplished in Ethernet networks using the Priority Flow Control (PFC)
mechanism~\cite{pfc}.  Using PFC, a switch can pause an incoming link when its
ingress buffer occupancy reaches a preset threshold. As long as sufficient
``headroom'' is reserved to buffer packets that are in flight during the time
takes for the PAUSE to take effect, no packet will be dropped due to buffer
overflow~\cite{cisco-whitepaper,dcqcn}. 

The PFC standard defines 8 classes, called priorities~\footnote{The word priority is a
misnomer. There is no implicit ordering among priorities -- they are really just
separate classes.}. Packets in each priority are buffered separately, and PAUSE
messages carry this priority.  When a packet arrives at port $i$ of switch $S$
with priority $j$, it is enqueued in queue $j$ of port $i$. If the queue length
now exceeds the PFC threshold, a pause message (XOFF) is sent to the upstream
switch connected to port $i$. The message carries priority $j$. The upstream
switch then stops sending packets with priority $j$ to switch $S$ on port $i$ until a resume
message (XOFF) with priority $j$ is received.


%% Since PAUsing is carried out
%% on a per-ingress port, and not on a per-flow basis, problems such as unfairness
%% and head-of-the-line blocking may occur~\cite{dcqcn}. The PFC standard defines 8
%% classes (called priorities), where packets in class are buffered separately, to
%% mitigate these problems~\cite{dcqcn}. However, sine each priority needs its own
%% dedicated headroom, typically, no more than two or three priorities are
%% used~\cite{rdmaatscale}.

{\bf Deadlock:} PFC prevents buffer overflow, but it can lead to deadlocks.
Deadlock forms when paused links form a cycle
(Figure~\ref{fig:deadlock_example}). Once formed, deadlock is ``permanent'' in
the sense that it will continue to exist even if no new traffic is injected into
the loop. Deadlocks in PFC-based networks (or more generally, in credit-flow
networks) are a well-known problem. It is not merely a theoretical problem -- it
has been reported in practice~\cite{rdmaatscale}.

It is well known that Circular Buffer Dependency (CBD) is a {\em necessary}
condition for deadlock formation~\cite{tcp-bolt,hu2016deadlocks}. {\em
Sufficient} condition for deadlock formation in PFC networks have yet to be
fully understood~\cite{hu2016deadlocks}. 

{\bf Prior work on deadlock avoidance:} Prior work on deadlock management falls
in two categories: deadlock avoidance, or deadlock detection and resolution. Our
focus in this paper is on deadlock avoidance.  Since {\em sufficient} conditions
for deadlock formation are not well characterized, deadlock avoidance schemes
focus on preventing CBD for occurring. This is done either by limiting or
modifying routing~\cite{tcpbolt} to avoid CBD, or by careful buffer
management~\cite{xxx}. 

However, these schemes fail to meet one or more of the three key challenges:
$(i)$ they cannot be deployed with existing routing, or, $(ii)$ they do not deal
with dynamic nature of data center networks, or, $(iii)$ they require excessive
switch buffers or number of priorities. 

We now describe these three challenges in more detail. See \S\ref{sec:related}
for a detailed review of prior work.

%% Prior work on deadlock formation fails to meet the three challanges: first, they
%% do not work with To avoid such deadlocks, {\em deadlock-free
%% routing}~\cite{tcpbolt} has been proposed. It guarantees that (if the routing
%% configuration is correct,) any traffic does not cause deadlock.
%% 
%% Unfortunately, achieving deadlock-free routing is inefficient, and may not even
%% be viable. Deadlock-free routing is achieved by eliminating Cyclic Buffer
%% Dependency (CBD)~\cite{deadlockfree}.  However, ensuring that there is never any
%% CBD is challenging.
%% 
%% First, deadlock-free routing largely limits the choice of topology. For example,
%% Stephens et al. \cite{tcpbolt} proposes to only use tree-based topology and
%% routing, and shows that it is deadlock-free.  However, there are a number of
%% other datacenter topologies and routing schemes that are not
%% tree-based~\cite{bcube, camcube, jellyfish}, and do not have deadlock-free
%% guarantee.
%% 
%% Second, due to bugs or misconfiguration, deadlock-free routing configuration may
%% turn into deadlock-vulnerable. In fact, recent work has observed a PFC deadlock
%% case in real-world tree-based datacenter\cite{rdmascale}, caused by the
%% (unexpected) flooding of lossless class traffic.  Furthermore, there are
%% multiple reports of routing loops due to misconfiguration in today's production
%% datacenters~\cite{everflow, libra}. If lossless traffic encounters any of these
%% loops, CBD is unavoidable.  
%% 
%% Indeed, a recent paper~\cite{hu2016deadlocks} argued that preventing CBD is
%% quite difficult, so instead we should focus on defining and preventing
%% ``simpler''{\em sufficient} conditions to avoid deadlock. 
%% 
%% In this paper, we show that it is indeed possible to prevent CBD, in any
%% topology, without any changes to the underlying routing protocol, using existing
%% data center hardware. 




