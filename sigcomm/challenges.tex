%\vspace{-0.1in}
\section{Challenges}
\label{sec:challenges}

\subsection{Working with existing routing protocols}
\label{sec:incremental}

Most of the current proposals~\cite{tcpbolt, karol2003prevention,
sancho2004,lash,wu2003fault} for deadlock avoidance are based on custom routing
protocols that go to great lengths to avoid CBD.  Unfortunately, this makes it
very difficult, if not impossible to deploy them in existing data center
networks. 

Modern data centers are built atop Ethernet and IP. Routing is accomplished in a
variety of ways, including BGP~\cite{vl2, facebookrouting} or using SDN-like
protocols~\cite{singh2015jupiter}.  No matter what routing protocols are used,
data center operators tune them carefully satisfy numerous requirements such as
manageability and fault tolerance.  In addition, operators invest heavily in
tools and technologies to monitor and maintain their networks.

Thus, it is very difficult for a data center operator to substantially change their routing
infrastructure in order to avoid deadlocks. This task appears especially
onerous because RoCEv2 itself can be deployed without any changes to routing
protocols -- RoCEv2 packets are encapsulated in normal UDP packets are routed
like any normal IP packets.

\subsection{Data Center Networks are Dynamic}\label{sec:reroute}

Building a deadlock avoidance scheme that does not make any changes to routing
protocols is easier said than done. The key problem is that the routing is
dynamic -- paths can change in response to link failures or load.

Figure~\ref{fig:basic_clos} shows a simplified (and small) version
of network deployed in our data center, and an example of up-down routing (also
called valley-free~\cite{qiu2007toward}).  In up-down routing, a packet first
goes UP from the source server to one of the common ancestor switches of the
source and destination servers, then it goes DOWN from the common ancestor to
the destination server.  In UP-DOWN routing, the following property holds: when
the packet is on its way UP, it should not go DOWN; when it is on its way DOWN,
it should not go UP. Thus, with up-down routing, there can be no CBD and hence
no deadlock.

However, packets may deviate from the UP-DOWN paths due to many reasons,
including link failures (leading to route flaps), which are quite common in data
center networks~\cite{netpilot,f10}.  As have shown
in~\cite{shpiner2016unlocking}, when the UP-DOWN property is broken and packets
may reroute multiple times between two layers of switches, and deadlocks may
form as a result, as shown in Figure~\ref{fig:priority_transition}(b).

Deviations from up-down routing are common in our data centers.  We
see hundreds of such routes per day, and they can persist for minutes or even
longer. Overall, we estimate that $0.001\%$ of the traffic is routed over such
paths. This may sound tiny, but given that our network carries exabytes of
traffic per day, the absolute amount of traffic affected by such routing is
in tens of petabytes. This makes the threat of deadlocks, as discussed
in\cite{rdmaatscale,shpiner2016unlocking,hu2016deadlocks} quite real.

\subsection{Limited Number of Lossless Queues}
\label{subsec:pfcheadroom}

One easy way to avoid deadlock without changing routing is to buffer packets
from each flow separately from other flows at each hop -- essentially putting
each flow in its own class, and applying per-hop backpressure only within the
class.  A more sophisticated scheme in~\cite{karol2003prevention} requires as
many priorities as the diameter of the network. 

One problem with this idea is that the PFC standard supports only 8 priority
classes. But modern commodity switches can realistically support fewer than 8
lossless classes.  The problem is that  to guarantee the lossless property, a
switch needs to reserve certain amount of {\it headroom} from the memory pool.
The size of the memory pool thus determines the number of lossless queues that
the switch ASIC can support.

The reserved headroom per port, per lossless queue is needed to absorb the
packets in flight from the time a receiver decides to send a PFC pause frame to
its upstream sender to the time the sender stops transmitting after receiving
the pause frame. We describe how the
headroom size is calculated in Appendix \ref{APPHEADROOM}.

From Appendix \ref{APPHEADROOM}, we can see that for a typical 32-port 40GbE
Ethernet switch, it needs to reserve 2.76MB memory as the headroom to support
one lossless queue. For a switch of 12MB memory, this is 23\% of the total
memory.

The headroom calculated in Appendix \ref{APPHEADROOM} ensures that  
packets are not dropped. In practice, the reservation size needs to be larger to 
ensure that the link is not under-utilized when the receiver un-pauses the
sender. Furthermore, we need to reserve buffers for non-RDMA (i.e. lossy) traffic, which is
still the dominating traffic in data centers. With these constrains, the
current commodity switches can typically support just two or three lossless
classes~\cite{rdmaatscale}.

New switching ASICs may be able to support more lossless queues by adding more
memory, using smaller cell size (64-byte) and by reducing the pause frame
response time. But even these are not expected to support more than four or five
lossless queues. Hence the solutions that require a large number of lossless
queues are not practical. 

We now describe how \sysname{} addresses these three challenges. 
