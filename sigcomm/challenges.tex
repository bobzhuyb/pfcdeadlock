%\vspace{-0.1in}
\section{Challenges}
\label{sec:challenges}

\subsection{Work with existing routing protocols and hardware}
\label{sec:incremental} Data center routing protocols have to satisfy a variety
of complex requirements regarding fault tolerance, and
security~\cite{beckett2016don}.  Operators also invest heavily in  tools and
technologies to monitor and maintain their networks; and these tools are
tailored for the routing protocols that are already deployed.  Thus, operators
are unwilling to deploy a brand-new routing
protocols like~\cite{dally,duato93,dally93,sancho2004,flich2012survey,lash,wu2003fault,glass,duato2001,domke2011,puente1999,dfedst16}
or hardware just for deadlock avoidance --  especially when RoCEv2 itself can be
deployed without any changes to routing\footnote{RoCEv2 packets are encapsulated
in standard UDP packets.}! 

\subsection{Data center networks are dynamic}\label{sec:reroute}

A deadlock avoidance scheme that works with existing routing infrastructure must
address the issue that most routing schemes are dynamic -- paths change in
response to link failures or other events.

Figure~\ref{fig:basic_clos} shows a simplified (and small) version
of network deployed in our data center, with commonly used up-down routing (also
called valley-free~\cite{qiu2007toward}) scheme.  In up-down routing, a packet first
goes UP from the source server to one of the common ancestor switches of the
source and destination servers, then it goes DOWN from the common ancestor to
the destination server.  In UP-DOWN routing, the following property holds: when
the packet is on its way UP, it should not go DOWN; when it is on its way DOWN,
it should not go UP. Thus, with up-down routing, there can be no CBD and hence
no deadlock.

However, packets can deviate from the UP-DOWN paths due to many reasons,
including link failures, port flaps etc., which are quite common in data
center networks~\cite{netpilot,f10}. When the up-down property is violated,
packets ``bouncing'' between layers can cause
deadlocks~\cite{shpiner2016unlocking}. See
Figure~\ref{fig:clos_1_bounce}.

In our data centers, we see hundreds of violations of up-down routing per
day. Such routes can persist for minutes or even longer. Overall, we estimate
that $0.001\%$ of the traffic is routed over such paths. This may sound tiny,
but given that our network carries exabytes of traffic per day, the absolute
volume of traffic affected by such routing is in tens of terabytes. This makes
the threat of deadlocks, as discussed
in\cite{rdmaatscale,shpiner2016unlocking,hu2016deadlocks} quite real.

\subsection{Limited number of lossless queues}
\label{subsec:pfcheadroom}

One idea to solve deadlock is to assign dynamic priority to packets. The
priority of a packet increases as the packet approaches its destination
~\cite{karol2003prevention}.  Such a design requires as many priorities as the
diameter of the network.  The problem with this idea is that the PFC standard
supports only 8 priority classes. Worse yet, commodity switches can
realistically support fewer than 8 lossless classes.  The problem is that  to
guarantee the lossless property, a switch needs to reserve certain amount of
{\it headroom} per port, per lossless queue.

The headroom is needed to absorb the packets that are in flight during the time
it takes for the PAUSE message to take effect.
The headroom size depends on many factors, including length
of cables. See Appendix \ref{APPHEADROOM} for details.  A
32-port 40GbE Ethernet switch needs to reserve 2.76MB of headroom to support one
lossless queue on all ports. A commodity switch like this typically has
12MB total buffer\footnote{The memory used to build switch buffers needs to be
extremely fast (e.g. 32x40GB switch needs memory that can support read/write
speeds of 1.28 Tbps), and hence is very expensive.}, so 23\%
of the total switch buffer is needed to support just one lossless priority.

While this headroom is sufficient to avoid packet drops, in practice, we use a
slightly larger value to avoid buffer under-flow when the receiver un-pauses the
sender.  Furthermore, we need to reserve buffers for non-RDMA (i.e. lossy)
traffic, which is still the dominating traffic in data centers. With these
constrains, the current commodity switches can typically support just two or
three lossless classes~\cite{rdmaatscale}.

New switching ASICs may be able to support more lossless queues by adding more
memory, using smaller cell size (64-byte) and by reducing the pause frame
response time. But even these are not expected to support more than four or five
lossless queues. Hence the solutions that require a large number of lossless
queues are not practical.

We now describe how \sysname{} addresses these three challenges.
