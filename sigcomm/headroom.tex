\subsection{Limited Number of Lossless Queues}\label{subsec:pfcheadroom}

The number of lossless queues that a switch can support is limited by two factors. First, the commodity switching ASICs typically support only a small number of queues (e.g., eight) and we need to use some of the queues for the lossy traffic. Second, to guarantee the lossless property, a switch needs to reserve certain amount of {\it headroom} from the memory pool. The size of the memory pool is of limited size, hence the number of lossless queues is further limited by the size of the memory pool.

The reserved headroom per port per lossless queue is to absorb the packets in flight from the time a receiver decides to send a PFC pause frame to its upstream sender to the time the sender stops transmitting after receiving the pause frame. See \cite{rdmaatscale} for how PFC works. We describe how the headroom size is calculated in Appendix \ref{APPHEADROOM}.

From Appendix \ref{APPHEADROOM}, we can see that for a typical 32-port 40GbE Ethernet switch, it needs to reserve 2.76MB memory as the headroom to support one lossless queue. For a switch of 12MB memory, this is 23\% of the total memory.

The headroom calculated in Appendix \ref{APPHEADROOM} is to make sure that packets cannot be dropped. In practice, the reservation size should be larger than that. This is because we need some additional reservation to make sure that the link is not under-utilized, when the receiver releases the sender from been paused. Furthermore, we need to reserve buffers for lossy traffic, which is still the dominating traffic in data centers. Consider all these constrains, the number of lossless queues can be supported by the current commodity switches typically is limited to two.

We note that new switching ASICs may be able to support more lossless queues by adding more memory, using smaller cell size (64-byte), reducing the pause frame response time. But the widely deployed switches support only two lossless queues. It is unlikely that the switches can support more than four or five lossless queues. Hence the solutions that use a large number of lossless queues are not practical. 



