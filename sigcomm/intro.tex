%\vspace{-0.1in}
\section{Introduction}\label{sec:intro}

\para{Datacenter appliations demand high performance from the network. RDMA provides such performance.
We deploy RDMA over Converged Ethernet (RoCE).} \\ 

\para{RoCE relies on PFC. PFC can cause deadlocks.} \\

\para{Prior work} proposes theoritical solutions. But none of them is practical for current datacenter deployment.
There are two main issues, 1) most of the solutions require significant changes on routing, enforcing or accurately predicting
the path of each packet. This is conflicted with the nature of Ethernet and IP networks and hard to achieve in practice.
We have evidences from production datacenters that a certain portion of packets are routed unexpectedly, though not dropped.
2) The number of lossless priority available is very limited. There are only several traffic classes in total,
and each lossless class requires siginifant reserved buffer to make PFC work. This is the biggest obstacle of 
deadlock free designs in practice.

\para{(Our solution)} In this paper, we work on a practical solution that can immediately implemented by 
commodity switches. We do not enforce the path of each packet or make any changes on the routing algorithm. 
This is important for network operators. We must also minimize the hardware resources usage, like number of priority
classes, reserved buffer and number of ACL rules. 

\para{(Limit the scope)} We do not discuss centralized SDN solutions that decide the path and timing of each flow. In 
addition, the centralized solution is not trivial (refer
to our hotnets paper). We assume our networks rely on PFC. Using centralized controller or getting rid of PFC would require 
significant changes in the network design, and may not have as good performance as RoCE.

\para{Specifically,} we seek to design an algorithm that takes the topology and routes that must be lossless as the 
input, and outputs the buffer configuration and ACL rules that guarantee no deadlock. Our key idea is to tag the packets along 
the path, and assign each packet to different lossless queues. Without any special assumptions about the topology, 
TTL is a type of tag that carries implicit information of what path the packet has took.
Properly configured, we can avoid cyclic buffer dependency using this information.
We design a greedy algorithm that combines multiple tags into one lossless priority and effectively reduce the number of 
required priorities.

\para{However, we found that this heurisitc does not always yield the optimal solution in terms of the number of lossless priorities. 
Therefore, we further investigate structured datacenter topology.} We show that, on Fat-tree and BCube, we can develop 
smarter tagging system than TTL and further reduce the number of lossless priorities.
In cases where the limit number of lossless queues cannot guarantee deadlock-free, we provide suggestions for 
network operator that which part of routing paths may be turned into lossy so that the whole network can be deadlock-free.


\para{We implement our algorithm and verify it using testbed and NS-3 simulations.} \\

\para{Our contributions.} 1) we analyze the practical challenges in making RoCE deadlock-free, including 
the challenges in enforcing/predicting routing and the limit of lossless queues. 2) we develop practical solutions
that eliminate deadlock for Fat-Tree and BCbue. 3) we extend our algorithm for generic graphs and scenarios
where the number of lossless queues is insufficient.


