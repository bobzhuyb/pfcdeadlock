\begin{appendices}
\section{PFC headroom calculation}\label{APPHEADROOM}

The PFC headroom needed per port per lossless queue can be calculated by
		considering the time interval needed for a receiver to pause its
		upstream sender. The time interval is composed of the following 6
		periods for the lossless class $p$:

	
\noindent\textbf{The time to send a PAUSE frame $t_1$}.  Once a pause frame is
		generated, it may be blocked by a packet that has just started
		transmision. Since Ethernet is non-preemptive, in the worst-cast,
		$t_1 = \frac{ L_{mtu} + L_{pfc}}{B}$, where $L_{mtu}$ is the MTU size,
		and $L_{pfc}$ is the size of a PFC pause frame, and $B$ is the link
		rate.

\noindent\textbf{The PAUSE frame propagation time $t_2$}. This depends on 
		the cable length between the sender and receiver.

\noindent\textbf{The PAUSE frame receiving time at the sender $t_3=\frac{L_{pfc}}{B}$}.

\noindent\textbf{The PFC response time $t_4$}. This is the amount of time needed
		for the sender to process the pause frame.

\noindent\textbf{The time for the sender to stop transmitting $t_5$}. Again,
		because Ethernet is non-preemptive, sender must finish 
		transmitting the packet that may have already started. Hence, in the
		worst case, $t_5 =
		\frac{L^{p}_{mtu}}{B}$.

\noindent\textbf{The time for the pipe to be drained $t_6$}. We know $t_6 =
		t_2$.


At a first glance, the headroom size should be $B\times\sum t_i$. But there are
some additional details. The switching ASICs typically divide a packet
into small cells of equal size for internal packet storage and
processing. The cell size ($C$) is typically larger than the smallest
Ethernet packet size (64 bytes). For one 64-byte packet, one cell is
allocated. So in the worst-case, the needed headroom size is:
\begin{eqnarray} \label{eqn:pfcheadroom} S_{hdr} & = &
C\lceil\frac{(t_1+t_2+t_3+t_4 + t_6)B}{64}\rceil + C\lceil \frac{t_5
B}{64}\rceil \nonumber \end{eqnarray}

For a typical 40GbE RoCEv2 setup, we have $L_{mtu}=1500$ bytes, $L_{pfc}=64$
bytes, $t_2=t_6=1us$ (for about 200 meters cable length), $t_4=2.75us$,
$L^{p}_{mtu}=1100$ bytes, $C=208$ bytes.  For a commodity switch with 32
full duplex 40GbE ports, the total headroom size needed is 2.76MB for
supporting one lossless queue.  

\section{Optimal tagging scheme for Clos}
\label{sec:clos_optimal}

Algorithm~\ref{alg:clos} produces optimal tagging for Clos networks. Note
similarity to Algorithm~\ref{alg:ttl}, except for the last step in the {\em for}
loop, which is specific to Clos networks.

\begin{algorithm}
	\small
    \KwIn{Clos topology and lossless routes $R$}
	\KwOut{A tagged graph $G(V, E)$}
	$V \gets Set()$\;
	$E \gets Set()$\; 
	\For{each path $r$ in $R$} {
		$tag \gets 0$\;
		\For{each hop $h$ in $r$} {
			$V \gets V \cup \{(h, tag)\}$\;
			$E \gets E \cup \{lastHop\rightarrow(h, tag)\}$\;
			\If{$h$ is going down \&\& nextHop is going up} {
				$tag \gets tag+1$\;
			}
		}
	}
	\Return{$G(V, E)$}\;
    \caption{The optimal tagging system for Clos topology.}
	\label{alg:clos}
\end{algorithm}
\end{appendices}
