%\vspace{-0.1in}
\section{Introduction}\label{sec:intro}

The growing demand for online services and cloud computing has driven today's
datacenter networks (DCNs) to a large scale with hundreds of thousands of
servers and thousands of switches. With this enormous number of network devices,
network failure and device upgrade become the norm rather than the
exception~\cite{zupdate, netpilot}. To maintain connectivity in face of
failures or device upgrades/updates, routing needs to reconfigured. It has been
well-recognized that updating large networks is not easy, especially if one
wants to ensure that there is no transient congestion~\cite{zupdate} or
blackholes~\cite{dionysus}.

At the same time, as DCNs enter the 40/100Gbps era, RDMA is currently being
deployed for achieving ultra-low latency, high throughput and low CPU
overhead~\cite{dcqcn,timely,rdmaatscale}.  To enable efficient operation,
RDMA usually runs over a  lossless L2 network, enabled by PFC~\cite{dcqcn}.

It is well known that use of PFC can lead to deadlocks~\cite{tcp-bolt}.  A
deadlock is a standstill situation where a set of switch buffers form a
permanent cyclic waiting dependency and no packet can get drained at any of
these buffers. Once deadlock occurs, no packet can be delivered through a part
of or even the whole DCN.

It was mistakenly assumed that tree-structured data center networks are not
subject to this problem. However, it has recently been demonstrated that this is
not the case~\cite{rdmaatscale}. Still, under static circumstances (i.e.,
when both of the network topology and the routing function are fixed), deadlocks
are rare, and can be avoided by using a routing scheme that ensures there is no
cycle in the buffer dependency graph~\cite{tcp-bolt}.

However, we have discovered that under dynamic circumstances (e.g. during
network reconfiguration), however, deadlock may occur during reconfiguration
process when transitioning from an old deadlock-free routing to a new
deadlock-free routing function. 

This is because during the routing reconfiguration process, due to the
asynchronous updates of switch rules, an arbitrary combination of old and new
paths may be in effect at the same time.  As has been shown
in~\cite{zupdate,dionysus}, this may induce congestion, loops etc. -- and it
may also induce a circular dependency among switch buffers. This can lead to
deadlock, which we refer to as \textit{reconfiguration-induced deadlock}.

Note that this is a far more serious problem that blackholes or congestion.
Those conditions are transient -- they dissipate as new routing take effect
fully. Deadlock, howver is {\em not} transient - once formed, new routing does
not alleviate it, since no packts are being transmitted among deadlocked
switches!

Reconfiguration-induced deadlock can be avoided by imposing some constraints on
the ordering of configuration actions during the reconfiguration process. For
example, given two deadlock-free routing matrices $R_{old}$ and $R_{new}$,
reconfiguration-induced deadlocks can be avoided by removing all paths in
$R_{old}$ and then setting up $R_{new}$. This, of course, is highly disruptive.
Ideally, we want to come up with a scheme that moves us from $R_{old}$ to
$R_{new}$ without ever creating a circular buffer dependency, while causing
minimal disruption to traffic. Typically, we want to complete the
reconfiguration as quickly as possible: i.e. in fewest number of steps. 

In this paper, we  develop an approach for achieving fast deadlock-free routing
reconfiguration as quickly as possible. It is based on two observations: 1)
there exist multiple  valid orderings that is deadlock-free; and 2) choosing an
ordering with minimum order dependencies among configuration actions can lead to
fast reconfiguration. Our approach is general and can be applied to arbitrary
DCNs, including Fat-tree, VL2, HyperX, Jellyfish, etc. 
