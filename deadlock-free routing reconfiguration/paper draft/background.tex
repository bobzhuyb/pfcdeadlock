\section{Background}\label{sec:background}

\subsection{PFC Deadlock Problem}\label{subsec:pfcdeadlock}

The deployment of RDMA over Ethernet requires Priority-based Flow Control (PFC)~\cite{pfc}  to provide a lossless L2 network~\cite{dcqcn, rdmaatscale}. 
PFC is a mechanism for ensuring zero packet loss under congestion in data center bridging (DCB) networks. When PFC is enabled, the switch will maintain a counter to track the virtual queue length of each ingress queue. Once the queue length reaches a pre-configured PFC threshold, a PAUSE frame will be generated to pause the incoming link for a specified period of time.

The using of PFC will cause PFC deadlock problem, as reported in ~\cite{rdmaatscale}. PFC deadlock arises when the buffer usage of a set of ingress queues reach their PFC thresholds simultaneously, and the paused links form a cycle. Once PFC deadlock is created, no packet is allowed to be transmitted through the affected links. 

Due to the backpressue paradigm of PFC~\cite{tcpbolt,dcqcn}, data transmission of the whole DCN may be paused by a PFC deadlock only among a small number of network devices. Hence it is important to avoid PFC deadlock in DCN.



\subsection{Cyclic Buffer Dependency}\label{subsec:cyclicbd}

Cyclic buffer dependency is a well-known necessary condition for deadlock problem~\cite{gerla1980flow}.

For a given DCN topology $\textbf{N}$ and a routing function $\textbf{R}$, we can construct the corresponding buffer dependency graph as follows. For any two ingress queues $RX_1$ and $RX_2$ in $\textbf{N}$, if there exists a path $p$ that traverses $RX_1$ and  $RX_2$ consecutively, we add a directed dependency edge from $RX_1$ to $RX_2$.  If there are cycles in the constructed buffer dependency graph, we say ($\textbf{N}$, $\textbf{R}$) has cyclic buffer dependency. PFC deadlock could occur when there is cyclic buffer dependency.

PFC deadlock can be avoided by adopting a deadlock-free routing function $\textbf{R}$ that will not create cyclic buffer dependency. Many algorithms have been proposed in the past for producing deadlock-free routing functions for a given network topology~\cite{dally,flich2012survey,tcpbolt}.

%\subsection{Classification of PFC Deadlock}\label{subsec:deadlockclassification}
%
%PFC Deadlock can be classified into 