%\vspace{-0.05in}
%\section{}\label{sec:model}
%\vspace{-0.05in}

Assuming $n_f$ greedy flows with infinite traffic demand are traversing the ingress queues of $n_s$ switches and $n_h$ hosts in a lossless network. Let $k$ be the port number of the switches (assuming all the switches have uniform port number). We logically consider a switch as k ingress queues, and a host as $1$ ingress queue. An ingress queue consists of $n^\prime$ ($n^\prime=k*n_s+n_h$) virtual output queues. For any ingress queue $q_i$, the queue length of its $j$-$th$ virtual output queue represents the bytes of packets to be sent to ingress queue $q_j$.

For any flow $f_j$ (j = 1,2,\ldots,$n_f$), we add a virtual ingress queue $q_{(j+n^\prime)}$ as the traffic source. Let $q_i$ be the first ingress queue flow $f_j$ traverses in the network. For $q_{(j+n^\prime)}$, the queue length of its $i$-$th$ virtual output queue is non-zero, and the others are all zero.

Let $\textbf{G}(\textbf{V},\textbf{E})$ be a directed graph. Let $n=n^\prime + n_f$.

\begin{enumerate}
\item Any vertex $v_i \in \textbf{V}$ (i = 1,2,\ldots,$n$) represents the $i$-$th$ ingress queue.

\item $\textbf{V}_0$ is a subset of $\textbf{V}$. Any vertex $v_i \in \textbf{V}_0$ (i = $n^\prime +1,n^\prime +2,\ldots,n$) represents the virtual ingress queue corresponding to the traffic source of $(i-n^\prime)$-$th$ flow.

\item Any arc $e_{i,j} \in \textbf{E}$ (i = 1,2,\ldots,$n$; j = 1,2,\ldots,$n$) means there exists a flow traversing ingress queues $i$ and $j$ in sequence.

\item Any arc $e_{i,j}$ in $\textbf{E}$ has a weight $w_{e_{i,j}}$ ($>=$0), which represents the queue length of the $j$-$th$ virtual output queue of ingress queue $i$.

\item $\forall v_i \in \textbf{V}$, we have $\sum_{j=1}^{n} w_{e_{i,j}} <= m_{\theta}$, where $ m_{\theta}$ is a constant. Here $ m_{\theta}$ represents the PFC threshold. This constraint is used to describe the fact that the queue length of any ingress queue can be no more than the PFC threshold.
\end{enumerate}

$\textbf{G}(\textbf{V},\textbf{E})$(t) can be viewed as a snapshot of the network state at time t. For any $e_{i,j}$ in $\textbf{E}$, $w_{e_{i,j}}(t)$ is the weight of $e_{i,j}$ at time t. we define

\begin{equation}\label{eq:1}
 \frac{d(w_{e_{i,j}}(t))}{dt} = \frac{w_{e_{i,j}}(t+\delta t)-w_{e_{i,j}}(t)}{dt}.
\end{equation}

The behavior of PFC pause can be defined as

\begin{equation}\label{eq:2}
 \frac{d(w_{e_{i,j}}(t))}{dt} >= 0 \text{ when } \sum_{k=1}^{n} w_{e_{j,k}}(t) = m_{\theta} .
\end{equation}

The queue length of an ingress queue cannot increase when all the virtual output queues pointing to it are empty, so we have

\begin{equation}\label{eq:3}
 \frac{d(w_{e_{i,j}}(t))}{dt} <= 0 \text{ when } \sum_{k=1}^{n} w_{e_{k,i}}(t) = 0.
\end{equation}

The queue length of an ingress queue should increase when not all the virtual output queues pointing to it are empty and PFC threshold is not reached, so we have

\begin{equation}\label{eq:4}
 \frac{d(w_{e_{i,j}}(t))}{dt} > 0 \text{ when } \sum_{k=1}^{n} w_{e_{k,i}}(t) > 0 \text{ and } \sum_{k=1}^{n} w_{e_{j,k}}(t) < m_{\theta}.
\end{equation}

The queue length of any ingress queue corresponding to the traffic source of any flow should be larger than zero, so we have

\begin{equation}\label{eq:5}
 \sum_{j=1}^{n} w_{e_{i,j}}(t) > 0 \text{ when } v_i \in \textbf{V}_0.
\end{equation}

\textbf{Definition of deadlock:} We say $\textbf{G}(\textbf{V},\textbf{E})$(t) has deadlock when $\exists t_0<\infty$ and $\exists \textbf{V}^\prime \subseteq \textbf{V}$,
\begin{enumerate}
 \item $\forall t > t_0$ and $\forall v_i \in \textbf{V}^\prime$, $\frac{d(w_{e_{i,j}}(t))}{dt}=0$ for any feasible j.
 \item $\forall t > t_0$ and $\forall v_i \in \textbf{V}^\prime$, $\sum_{j=1}^{n} w_{e_{i,j}}(t) > 0$.
\end{enumerate}

\textbf{Sufficient and necessary condition for deadlock:} If $\exists t_1<\infty$ and $\exists \textbf{V}^\prime \subseteq \textbf{V}$,
 \begin{enumerate}
\item $\forall v_i \in \textbf{V}^\prime$,  $\sum_{j=1}^{n} w_{e_{i,j}}(t_1) = m_{\theta}$.

\item $\forall v_i \in \textbf{V}^\prime$ and $\forall v_j \in \textbf{V}-\textbf{V}^\prime$, there is either no edge between $v_i$ and $v_j$, or $w_{e_{i,j}}(t_1)=0$.
\end{enumerate}

Then $\textbf{G}(\textbf{V},\textbf{E})$(t) has deadlock.

\textbf{Proof:}

$\Rightarrow$ As $\forall v_i \in \textbf{V}^\prime$,  $\sum_{j=1}^{n} w_{e_{i,j}}(t_1) = m_{\theta}$, according to equation (\ref{eq:2}), if $v_i,v_j \in \textbf{V}^\prime$, $\frac{d(w_{e_{i,j}}(t_1))}{dt} >= 0$. However, as we have the constraint $\sum_{j=1}^{n} w_{e_{i,j}}(t_1) <= m_{\theta}$, we have $\frac{d(w_{e_{i,j}}(t_1))}{dt} = 0$.

As $\forall v_i \in \textbf{V}^\prime$ and $\forall v_j \in \textbf{V}-\textbf{V}^\prime$, there is either no edge between $v_i$ and $v_j$, or $w_{e_{i,j}}(t_1)=0$, according to equation (\ref{eq:3}), if $v_i \in \textbf{V}^\prime$, and $v_j \in \textbf{V}-\textbf{V}^\prime$, $\frac{d(w_{e_{i,j}}(t_1))}{dt} <= 0$. As $w_{e_{i,j}}(t_1)>=0$, we have $\frac{d(w_{e_{i,j}}(t_1))}{dt} = 0$

In summary, we have $\forall v_i \in \textbf{V}^\prime$, $\frac{d(w_{e_{i,j}}(t))}{dt}=0$ for any feasible j. It is trivial that $\forall v_i \in \textbf{V}^\prime$, $\sum_{j=1}^{n} w_{e_{i,j}}(t) > 0$. As the two constraints still hold at time $t+\delta t$, we can then prove $\forall t >= t_1$, $\frac{d(w_{e_{i,j}}(t))}{dt}=0$ is true for any $v_i \in \textbf{V}^\prime$. So $\textbf{G}(\textbf{V},\textbf{E})$(t) has deadlock.

$\Leftarrow$ (not finished yet) The key of the proof is to derive a $\textbf{V}^\star$ from $\textbf{V}^\prime$ which satisfies the two constraints of the sufficient and necessary condition. I need more time to work out the details.


%Let $t_1$ be a time that satisfies $t_1 > t_0$ and $t_1<\infty$. Assuming $\exists v_i \in \textbf{V}^\prime$, $\sum_{j=1}^{n} w_{e_{i,j}}(t_1) < m_{\theta}$, then the first constraint will be violated. So we have $\sum_{j=1}^{n} w_{e_{i,j}}(t_1) = m_{\theta}$.




%As $ \forall v_i \in \textbf{V}\prime$, and $\forall v_j \in \textbf{V}$, $ \frac{d(w_{e_{i,j}}(t_1))}{dt} = \frac{w_{e_{i,j}}(t_1+\delta t)-w_{e_{i,j}}(t_1)}{dt} = 0$, we have $w_{e_{i,j}}(t_1+\delta t)=w_{e_{i,j}}(t_1)$. According to equation (\ref{eq:4}), either $\sum_{k=1}^{n} w_{e_{k,i}}(t_1) = 0$ or $\sum_{k=1}^{n} w_{e_{j,k}}(t_1) = m_{\theta}$.
%
%As $\forall v_i \in \textbf{V}\prime$, $\sum_{j=1}^{n} w_{e_{i,j}}(t_1) > 0$, if $\sum_{k=1}^{n} w_{e_{k,i}}(t_1) = 0$ and $\sum_{k=1}^{n} w_{e_{j,k}}(t_1) < m_{\theta}$, according to equation (\ref{eq:4}), we have \frac{d(w_{e_{i,j}}(t_1))}{dt} < 0



