\abstract{
Driven by the need for ultra-low latency, high throughput and low CPU overhead,
Remote Direct Memory Access (RDMA) is being deployed by many cloud providers.
To deploy RDMA in Ethernet networks, Priority-based Flow Control (PFC) must be
used. PFC, however, makes Ethernet networks prone to deadlocks. Prior work on
deadlock avoidance has focused on {\em necessary} condition for deadlock
formation, which leads to rather onerous and expensive solutions for deadlock
avoidance. In this paper, we investigate {\em sufficient} conditions for
deadlock formation, conjecturing that avoiding {\em sufficient} conditions might
be less onerous.
}



%In this paper, we study the deadlock problem in lossless datacenter networks. Cyclic buffer dependency is a well-known necessary condition for deadlock. Most of prior solutions for avoiding deadlock is to eliminate cyclic buffer dependency in the network. Though obeying this principle can guarantee a deadlock-free network, its cost can be very expensive. With case study about several representative deadlock cases, we have found that there are some cases where cyclic buffer dependency is met, but there is no deadlock.  This finding indicates that prior solutions are designed based on a too strict condition, and thus introduces some unnecessary overhead. Motivated by this, we have also designed some preliminary solutions for deadlock avoidance where breaking cyclic buffer dependency is not a must.