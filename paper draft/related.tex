%\secspacelarge
\section{Related Work}
%\secspace

\para{RDMA in datacenters.} RDMA has been used for improving distributed
application performance, like in-memory key value store~\cite{mitchell13atc,
farm, kalia14sigcomm}, Hadoop RPC~\cite{hadooprdma} and HBase~\cite{hbaserdma}.
It has been recently deployed inside modern
datacenters~\cite{timely,dcqcn,rdmascale}, based on RoCE (RDMA over Converged
Ethernet), which relies on PFC to create a lossless Ethernet.  Recent
work~\cite{timely,dcqcn} discusses congestion control for RoCEv2
networks. The issue of deadlock is mentioned in these papers, but not directly
addressed. In this paper, we aim to get deeper understanding on deadlock and
possible mitigations.

\para{Deadlock-free routing.} To avoid deadlock in lossless networks, previous
work~\cite{tcpbolt,karol2003prevention,lash,sancho2004,wu2003fault} has focused
on deadlock-free routing: i.e. deadlock freedom regardless of traffic pattern
etc. It has also been proven that that eliminating cyclic buffer dependency is a
necessary and sufficient condition for deadlock-free
routing~\cite{deadlockfree}. However, deadlock-free routing is difficult to
implement in practice -- since it is challenging to eliminate cyclic buffer
dependency in face of arbitrary bugs and failures. Our work explores how we may
control the flows, packet formats and switch configurations to avoid deadlock
even if routing is not deadlock-free.






