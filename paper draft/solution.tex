\section{Potential Deadlock Mitigations}
\label{sec:mitigation}

Since cyclic buffer dependency is a loose condition for deadlock, there are 
mitigation mechanisms that avoid deadlock even if cyclic buffer dependency is present. 
The examples and analysis in Section~\ref{sec:analysis}, {\em e.g.,} Equation~\ref{eq:condition},
inspire us with some of the following potential deadlock mitigations. 


\para{TTL-based mitigation for deadlock caused by loops.} Upon a routing loop, TTL plays an important role in 
determining whether deadlock is created. The smaller TTL, the less possible deadlock forms.
Thus, the most straightforward mitigation is to reduce packets' initial TTL values.
For example, with an $N$-hop routing loop, if the initial TTL is not larger than $N$,
no deadlock will form because the deadlock threshold for $r$ is $B$, as shown in 
Equation~\ref{eq:condition}.

In practice, we may not be able to guarantee that initial TTL values are always smaller than
the size of the loop. However, by proper switch buffer management, we may make {\em class-specific}
TTL much smaller than the actual TTL values. For example, if we assign packets with TTL that 
is different by at least $X$ to different priority classes, the effective TTL becomes $X$
within a priority class. Since PFC PAUSE operates based on priority classes, the deadlock threshold
of injecting rate $r$ is effectively increased.

In worst case scenarios, the effective TTL may still be larger than the size of loop, meaning
that some $r$ smaller than $B$ leads to deadlock. We may consider rate limiting to keep 
$r$ below the threshold $NB/TTL$.

\para{Rate limiting.} Commodity switches support bandwidth shaping for each priority class
or even particular flows. This can mitigate deadlock caused by both routing loops and multi-flow
buffer dependency. \fixme{more here.}


\para{Preventing PFC from propagating to upper tier.} 
In Clos networks, a deadlock that involves high tiers can cause the most serious damage.
A natural mitigation is to limit the impact of PFC locally near servers. This reduces
the risk of deadlock as well as the impact once deadlock occurs.
The idea is to assign different PFC thresholds to different tiers and make PFC happen less
on higher tier. Also, higher tier switches may be allowed to ignore PFC or drop packets from
lower tier upon extreme cases.
The price is that this turns the network into partly
``lossy'' and may cause congestion drops.



In future work, a deeper understanding of tighter conditions for deadlock may
lead to more deadlock mitigations.