\section{Future Work: Deadlock Mitigations}
\label{sec:mitigation}

Since cyclic buffer dependency is a loose condition for deadlock, there are
mitigation mechanisms that avoid deadlock even if cyclic buffer dependency is present.
The examples and analysis in Section~\ref{sec:analysis}
inspire us with some of the following potential deadlock mitigations.


\para{TTL-based mitigation for deadlock caused by loops.} In a routing loop, 
the smaller TTL, the less possible deadlock forms (see Equation~\ref{eq:condition}).
Thus, the most straightforward mitigation is to reduce packets' initial TTL values.
For example, with an $N$-hop routing loop, if the initial TTL is not larger than $N$,
no deadlock will form because the deadlock threshold for $r$ is $B$. The flow throughput
can never exceed the threshold.

In practice, we may not be able to guarantee that initial TTL values are always smaller than
the size of the loop. However, by proper switch buffer management, we may make {\em class-specific}
TTL much smaller than the actual TTL values. For example, if we assign packets 
that have different TTL values by at least $X$ to different priority classes, the effective TTL becomes $X$
within a priority class. Since PFC PAUSE operates based on priority classes, the deadlock threshold
of injecting rate $r$ is effectively increased.

In worst case scenarios, the effective TTL may still be larger than the size of loop, meaning
that some $r$ smaller than $B$ leads to deadlock. We may consider rate limiting to keep
$r$ below the threshold $NB/TTL$, as discussed below.

\para{Rate limiting.} Commodity switches support bandwidth shaping for each priority class
or even particular flows. This can mitigate deadlock caused by both routing loops and multi-flow
buffer dependency, as shown in Section~\ref{sec:analysis}. If we are able to predict the rate 
threshold for deadlock, we may bound the individual flow rate by that threshold on switches 
that are involved in cyclic buffer dependency. However, this requires intelligent rate
limiting schemes to avoid over-punishing innocent flows. We leave this to future work.


\para{Limiting PFC pause frames near the source}
PFC is well known for its HoL blocking problem. The damage of HoL and the potential deadlock caused 
by PFC is significant because the pause frames are generated near the destination or in the middle of 
the network, where network congestions usually happen. Hence if we can limit the PFC pause frame 
generation near the source, we can reduce the damage of both deadlock and HoL blocking. 

Here we describe several possible ways of doing so: first, we can assign different PFC thresholds to 
the ports of a switch based on their positions. Ports connecting to the downstream get smaller threshold, 
whereas ports connecting to the upstream get larger threshold. 
Second, we can use switches with larger threshold values at the higher tiers so that they can absorb 
temporal burstiness instead of generating PFC pause frames. Third, again, we may classify 
packets with different TTL into different classes and assign them different PFC thresholds.
Unfortunately, these solutions may lead to other issues including the unfairness between long (across 
different high tier switches) and short ({\em e.g.,} within the same rack) flows. 
This trade-off is less understood and worthy of further study.

% using larger threshold values is tricky, 
%as it does not push PFC back to the source for all cases.)

%In Clos networks, a deadlock that involves high tiers can cause the most serious damage.
%A natural mitigation is to limit the impact of PFC locally near servers. This reduces
%the risk of deadlock as well as the impact once deadlock occurs.
%The idea is to assign different PFC thresholds to different tiers and make PFC happen less
%on higher tier. Also, higher tier switches may be allowed to ignore PFC or drop packets from
%lower tier upon extreme cases.
%The price is that this turns the network into partly
%``lossy'' and may cause congestion drops.

%Second, we may classify packets based on the TTL value they carry, and use TTL-based buffer 
%management similar to that in \ref{karol2003prevention}. The smaller the TTL value of a packet, 
%the smaller the chance that the packet will trigger PFC pause frames. Hence the TTL-based buffer 
%management has the ability to limiting the pause frame generation near the source. But its fairness 
%property is less understood and worthy of further study.

\para{Preventing PFC from been generated.}
The recent transport protocols, DCQCN~\cite{dcqcn} and TIMELY~\cite{timely} are designed to reduce 
the possibility of PFC generation. But due to the feedback 
latency introduced by the end-to-end delay, and the fact that both algorithms react only after the 
congestions been generated (through either queue length or latency increase), they cannot prevent PFC 
from been generated. 

One possible way to further preventing PFC from been generated is to integrate DCQCN together with 
phantom queuing. By reacting to the phantom queues which assume lower link speed, DCQCN can generate 
congestion signals much earlier. 

\para{Other possible mechanisms.} In future work, a deeper understanding of tighter conditions 
for deadlock may lead to more deadlock mitigation mechanisms. In future work, we are going to 
investigate the tightest possible conditions -- necessary and sufficient condition for deadlock. 