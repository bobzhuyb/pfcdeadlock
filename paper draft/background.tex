\secspace
\section{Deadlock in Lossless Network}
\secspace

In this section, we explain the background of deadlock in lossless network, and 
why we must study the necessary and sufficient condition of it.

\para{Lossless Ethernet relies on PFC.} 
Modern data center networks require efficient network stacks in order to support 
40Gbps or even higher bandwidth. RDMA over Ethernet, or RoCE, is one of the most popular
solutions and has been deployed in both Google and Microsoft data centers~\cite{dcqcn, timely}.
The key property of RoCE is that it runs on lossless Ethernet, meaning
that no packet is dropped due to congestion. This is achieved by enabling PFC 
(Priority-based Flow Control).

PFC introduces PAUSE mechanism, with which a switch can pause an incoming link
when its ingress buffer occupancy reaches a preset threshold. Properly tuned, the 
buffer will never be full, and no packet will be dropped due to full buffer.
As shown below, deadlock may occur in such lossless networks.


\para{Deadlock.}
In this paper, we define deadlock as a permanent state in which a subset of links are all paused
by PFC, and none of the packets in the involved buffers can ever move even there is 
no more packet being sent into these links.

\para{PFC may lead to deadlock, if paused links form a circle.}
In a PFC-enabled network, if a subset of link congests and the corresponding switch queue lengths reach
the PFC threshold at the same time, the links are all paused. If these links happen
to form a complete circle, the network has a deadlock. No packet can move because 
the links are paused, even if there is no more new traffic injected into this circle.

This essentially means {\em cyclic buffer dependency}. Every link is waiting for its next
hop to RESUME the link, while this depends on the buffer status of the next hop. However,
the next hop buffer depends on the further next hop and wait for its RESUME,
so that it can send out packets to lower the buffer usage. This goes on, and in the end, 
any stuck buffer depends on itself to move, thus forming a deadlock.

To avoid deadlock, previous work has considered {\em deadlock-free routing}. Deadlock-free
routing guarantees that (if the routing configuration is correct,) any traffic does not
cause deadlock.

\para{Unfortunately, achieving deadlock-free routing has high price, and may not even be viable.} 
In previous work, to achieve deadlock-free routing is to completely avoid cyclic buffer dependency.
It is proved that cyclic buffer dependency-free is necessary and sufficient to deadlock-free routing.
However, ensuring there is always no cyclic buffer dependency is challenging.

First, deadlock-free routing largely limits the choice of topology. For example, \cite{tcpbolt}
proposes to only use tree-based datacenter topology and routing, and shows that it is deadlock-free. 
However, tree-based topology and routing has well-known problems including oversubscription~\cite{fattree},
large path length~\cite{jellyfish} and vulnerable to failures.
Thus, there are a number of other datacenter topology and routing schemes that are not 
tree-based~\cite{bcube, camcube, jellyfish} and do not have deadlock-free guarantee. 

Second, upon bugs or misconfiguration, deadlock-free routing configuration may turn into
deadlock-vulnerable. In fact, recent work has observed a PFC deadlock case in production
datacenter environments\cite{rdmascale}. It is caused by (unexpected) flooding of 
lossless class traffic. Furthermore, there are multiple reports of routing loops in today's non-RDMA
production datacenters~\cite{everflow, libra}, due to misconfiguration. If lossless traffic encounters
any of these loops, cyclic buffer dependency is unavoidable and deadlock can form in milliseconds.

In this paper, we argue that we accept the fact that cyclic buffer dependency may not be completely 
avoided,\footnote{In other words, deadlock-free routing may not always apply or work correctly.}
but try to understand more precise conditions of deadlock. Our findings show that,
even if there is cyclic buffer dependency, deadlock may not occur (see Section~\ref{sec:analysis}).
This means that cyclic buffer dependency is a quite loose condition for deadlock.
In this paper, we show our work in progress towards a tighter, and ideally a necessary and sufficient 
condition for deadlock. Once we have better understanding, we may avoid deadlock with smaller price. 
See discussion in Section~\ref{sec:mitigation}.


