\section{Deadlock in Lossless Network}

In this section, we explain the background of deadlock in lossless network, and 
why we must study the necessary and sufficient condition of it.

\para{Lossless Ethernet relies on PFC.} 
Modern data center networks require efficient network stacks in order to support 
40Gbps or even higher bandwidth. RDMA over Ethernet, or RoCE, is the most popular
solution and has been deployed in both Google and Microsoft data centers~\cite{dcqcn, timely}.
The key property of such network is that it runs on lossless Ethernet, meaning
that no packet is dropped due to congestion. This is realized by PFC 
(Priority-based Flow Control).

PFC introduces PAUSE mechanism, with which the switch can pause an incoming link
when its ingress buffer occupancy reaches a preset threshold. Properly tuned, the 
buffer will never overflow, and no packet will be dropped due to full buffer.
As shown below, deadlock may occur in such lossless networks.


\para{Deadlock.}
In this paper, we define deadlock as a permanent state in which a subset of links are all paused
by PFC, and none of the packets in the involved buffers can ever move even there is 
no more packet being sent into these links.

\para{PFC may lead to deadlock, if PFC-paused links form a circle.}
In a PFC-enabled network, if a subset of links congest and the buffer usage reach
the PFC threshold at the same time, the links are all paused. If these links happen
to form a complete circle, the network has a deadlock. No packet can moves because 
the links are paused, even if there is no more new traffic injected into this circle.

This essentially means {\em cyclic buffer dependency}. Any one link is waiting for its next
hop to RESUME the link, while this depends on the buffer status of the next hop. However,
the next hop buffer depend on the next next hop and wait for the RESUME
so that it can send out packets to lower the buffer usage. This goes on, and in the end, 
any stuck buffer depends on itself to move, thus forming a deadlock.

To avoid deadlock, previous work has considered {\em deadlock-free routing}.

\para{Achieving deadlock-free routing has high price, and may not even be viable.} 
In previous work, to achieve deadlock-free routing is to completely avoid cyclic buffer dependency.
It is proved that cyclic buffer dependency-free is necessary and sufficient to deadlock-free.
However, ensuring there is always no cyclic buffer dependency is challenging.

First, deadlock-free routing largely limits the choice of topology. For example, TCP-bolt 
has lengthy discussion on the design of deadlock-free routing and propose certain topology and routing. 
In today's datacenter network, a common configuration is to only use Clos-style topology, which does 
not have cyclic buffer dependency problem when the topology is complete. However, a number
of other topology setups are not cyclic buffer dependency-free.

Second, upon bugs or misconfiguration, deadlock-free routing configuration may become
vulnerable to deadlock. In fact, we have observed a deadlock case in our Clos network
deployment with PFC. \fixme{example here}


In this paper, we argue that we do not need to completely avoid cyclic buffer dependency.
Even if there is cyclic buffer dependency, deadlock may not occur (see Section~\ref{sec:analysis}).
This means that cyclic buffer dependency is a quite loose condition for deadlock.
In this paper, we aim to explore a tighter condition, ideally a necessary and sufficient condition for deadlock.
Once we have better understanding, we may avoid deadlock with smaller price ({\em e.g.,} 
Section~\ref{sec:mitigation}).


