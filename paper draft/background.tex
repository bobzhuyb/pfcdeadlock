\section{Deadlock in Lossless Network}

\para{RDMA over Ethernet relies on PFC.}

\para{PFC may lead to deadlock, if forming a cycle of PFC-paused links.}

We define deadlock as a permanent state in which a subset of links are all paused
by PFC, and none of the packets in the involved buffers can ever move even there is 
no more packet being sent into these links.

\para{Achieving deadlock-free routing has high price, and may not even be viable.} 
The common way to achieve deadlock-free routing is to avoid cyclic buffer dependency.
It is proved that cyclic buffer dependency-free is necessary and sufficient to deadlock-free.
However, ensuring there is always no cyclic buffer dependency is challenging.

First, deadlock-free routing largely limits the choice of topology. For example, TCP-bolt 
has lengthy discussion on the design of deadlock-free routing and propose certain topology and routing. 
In today's datacenter network, a common configuration is to only use Clos-style topology, which does 
not have cyclic buffer dependency problem when the topology is complete. However, a number
of other topology setups are not cyclic buffer dependency-free.


Second, upon bugs or misconfiguration, deadlock-free routing configuration may become
vulnerable to deadlock. In fact, we have observed a deadlock case in our Clos network
deployment with PFC. (example here)


