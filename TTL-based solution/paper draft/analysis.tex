%\vspace{-0.1in}
\section{Buffer Analysis}\label{sec:sol}

\subsection{PFC Headroom}\label{subsec:pfcheadroom}


A PAUSE message sent from a receiver to an upstream sender needs some time to arrive and take effect. To avoid packet drops, the receiver (i.e., the sender of the PAUSE message) must reserve enough buffer to accommodate any packets it may receive during this time. In this part, we calculate the amout of buffer needed as the PFC headroom.

\textbf{ Per port per priority PFC headroom:} At first, we calculate the time for a PFC message to arrive and take effect at its destination port. It mainly consists of six parts.

\begin{enumerate}
	
\item\textbf{The time to send a PAUSE message at the receiver side (denoted as $t_{snd}$)}: The transmission of the PAUSE frame  can pass ahead of any other packet queued in the receiver, but cannot preempt another frame currently being transmitted in the same direction. Hence in the worst case, the receiver generates a PAUSE frame right when the first bit of a maximum-size packet (i.e., the size is equal to Maximum transmission unit (MTU) ) has started engaging the transmission logic. So we have  $t_{snd}=(s_{MTU}+s_{PFC})/r_{l}$, where $s_{MTU}$ is the value of MTU, $s_{PFC}$ is the size of PFC message and $r_{l}$ is the line rate of the network link.

\item\textbf{The time for the PAUSE message to propagate from the receiver to the sender over the network link (denoted as $t_{wire}$)}: The value of  $t_{wire}$ is related to the materia and the length of the network links in use.

\item\textbf{The time to receive a PAUSE message at the sender side (denoted as $t_{rev}$)}: It is easy to know $t_{rev}=s_{PFC}/r_{l}$.

\item\textbf{The time to process a PAUSE message at the sender side (denoted as $t_{pro}$)}:  After a PAUSE message has been received by the sender, it will take an implementation-dependent amount of time to process the message and stop packet transmission.

\item\textbf{The time to stop packet transmission at the sender side (denoted as $t_{stop}$)}: After the sender finally decides to stop packet transmission, it can stop only at packet boundaries, to avoid packet corruption.  In the worst case, the sender will have completed the process of the PAUSE message just when the first bit of a maximum-size packet has started engaging the packet transmission. So we have  $t_{snd}=s_{MTU}/r_{l}$

\item\textbf{The time for the remaining bytes of packets on the link to get drained after stopping packet transmission at the sender(also denoted as $t_{wire}$)}: This part of time is equal to the time for the PAUSE message to propagate from the receiver to the sender.
\end{enumerate}

In summary, the per port per priority PFC headroom can be expressed using the following equation:


\begin{eqnarray} \label{eqn:pfcheadroom}
b_{hr} &=& r_{l}*(t_{snd}+t_{wire}+t_{rev}+t_{pro}+t_{stop}+t_{wire})     \nonumber \\
&=& 2(s_{MTU}+s_{PFC}+r_l*t_{wire})+r_l*t_{pro}
\end{eqnarray}

For typical TCP/IP based RDMA DCNs, we have $s_{MTU}=1500$bytes, $s_{PFC}=64$bytes. The length of links used in a single DCN is usually no larger than 300 meters~\cite{rdmaatscale}. As every 100 meters of link delays the reception of a packet by about 500ns for copper cables~\cite{pfcheadroom}, we have $t_{wire} \leq 1.5us$.  

The value of  $t_{pro}$ is implementation related. Let a quanta be the time needed to transmit 512 bits at the current network speed. The PFC definition caps this time to 60 quanta for any implementation~\cite{pfcheadroom}. Hence we have $r_l*t_{pro}=512 * 60 = 30,720 $ bits.

According to the above parameters, when  $r_l=40Gbps$, the per port per priority PFC headroom $b_{hr} \leq 21968$ bytes $\approx 22$ KB.

\textbf{PFC headroom of a switch:} For a $n$ port switch which supports $k$ priority classes, PFC PAUSE is possible to be triggered at all ports and priority classes simultaneously. So we should at least reserve $n*k*b_{hr}$ buffers as PFC headroom at every switch. 

For commodity switches like Arista 7050QX32 which has 32 full duplex 40Gbps ports, when supporting 8 priority classes, it requires about $32*8*22=5632$KB buffer per prioriry as the PFC headroom.

\subsection{Necessary Buffer for Achieving Work-Conservation}\label{subsec:bufferthroughput}

According to the rule-of-thumb~\cite{crtt,sizingrouterbuffer}, we at least need a buffer of size $b = C*RTT$ per port for achieving work-conservation, where RTT is the average round-trip time, and C is the link capacity. As our TTL-based solution allocates dedicated buffer to different priority classes, for a switch of n ports, when supporting $k$ priority classes, it demands a buffer of size $n*k*C*RTT$. 

Assuming that the average RTT in the DCN is $50$us. For Arista 7050QX32 switch, when supporting 8 priority classes, we need to reserve a buffer of size $32*8*40Gbps*50us=64MB$. This apparently exceeds the total buffer size of most commodity switches. The key to solving this problem is to let packets of different priority classes share some buffer.

On the other hand, we also need to ensure that PFC threshold is not reached before the switch has a chance to mark packets with ECN. This has been well discussed in DCQCN~\cite{dcqcn}. Basically, we can leverage a dynamic PFC thresholding scheme to dynamically share the switch buffer among different ports.

\subsection{Preliminary Solution For Reducing Buffer Demand}\label{subsec:solforreducebuffer}

\textbf{Reducing the PFC headroom}: For tree-based topologies like Fat-tree, packets of consecutive priority classes will not enter the same switch. So the number of priority classes a switch needs to support is halved. In addition, if we enforce that every switch drops the packets bounced back by the immediate downstream devices, a switch needs to support only $1/4$ of the total priority classes in use.

\textbf{Reducing the necessary buffer for achieving work-conservation}: The main problem here is that our TTL-based solution allocates dedicated buffer to different priority classes. To solve this problem, instead of simply allocating dedicated buffer to different priority classes, we can let packets of different priority classes to have both dedicated and shared buffer.

Specifically, we divide the switch buffer into $k+1$ partitions. The first $k$ partitions are dedicated for the k priority classes, correspondingly. The $k+1$ partition is shared by all the priority classes. If a packet is classified as priority class $i$, at first the switch will check whether partion $i$ has enough spare buffer to accommodate this packet. If yes, this packet will be placed in partition $i$. Otherwise, the switch will place this packet in partition $k+1$.

The original dynamic PFC thresholding scheme cannot be directly applied under the new buffer allocation scheme. Hence we modify it as follows. Let $b_i$ be the size of partition $i$, $t^i_{PFC}$ be the PFC threshold for priority class $i$, $s_i$ be the amount of buffer of partition $i$ that is currently occupied. We have

\begin{eqnarray} \label{eqn:pfcthreshold}
t^i_{PFC} &=& \alpha\frac{b_i}{\sum_{j=1}^k b_j} (b_{k+1}-s_{k+1}) + \alpha (b_i-s_i)
\end{eqnarray}

The intuition behind Equation~\ref{eqn:pfcthreshold} is to share the buffer of partition $k+1$ among different priority classes with respect to the weight $\frac{b_i}{\sum_{j=1}^k b_j}$. As for the buffer of dedicated partitions, we follow the idea of original dynamic PFC thresholding scheme to dynamically share it among different ports within the same priority class.

Under the new buffer allocation scheme and the new dynamic PFC thresholding scheme, the buffer needed for achieving work-conservation is reduced to 
$n*C*RTT$, which is within the buffer capacity of most commodity switches.
%\textbf{ Case of multiple priority classes:} When $k$ priority classes are enabled simultaneously at a single port, we cannot simply calculate the PFC headroom of a switch as $k*b_{hr}$. The reason is that in this case packets of any priority class cannot be transmitted at full line rate as the bandwidth is now shared by $k$ priority classes. Let $r_i$ be the packet transmission rate of priority class $i$, $1\leq i \leq k$. The per port PFC headroom of priority class $i$ can be expressed using the following equation
%
%\begin{eqnarray} \label{eqn:mpriority}
%b^i_{hr} &=& r_{i}*(t_{snd}+t_{wire}+t_{rev}+t_{pro}+t_{stop}+t_{wire})     \nonumber \\
%&=& 2(s_{MTU}+s_{PFC}+r_i*t_{wire})+r_i*t_{pro}
%\end{eqnarray}
%
%Then the total per port PFC headroom of all the $k$ priority classes can be calculated as 
%
%\begin{eqnarray} \label{eqn:sumpriority}
%b_{hr}&=&\sum\limits_{i=1}^k b^i_{hr}       \nonumber \\
%&=& 2k(s_{MTU}+s_{PFC})+ 2\sum\limits_{i=1}^k r_i*t_{wire}+\sum\limits_{i=1}^kr_i*t_{pro}     \nonumber \\
%&=& 2k(s_{MTU}+s_{PFC})+ 2r_l*t_{wire}+r_l*t_{pro}
%\end{eqnarray}
